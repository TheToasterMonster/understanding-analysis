\documentclass[12pt, letterpaper]{article}
\usepackage[margin=1in]{geometry}
\usepackage{enumitem}
\usepackage{graphicx}
\usepackage{amsmath}
\usepackage{amssymb}
\usepackage{amsthm}
\usepackage{bbm}
\usepackage{bm}
\usepackage{xcolor}
\usepackage{soul}
\usepackage{adjustbox}
\usepackage{titling}
\usepackage{parskip}
\usepackage{tcolorbox}

\graphicspath{ {./images/} }

% \setlist[enumerate, 2]{label=\roman*.}
\setlist[enumerate]{nosep}
\setlist[enumerate]{topsep=0pt}

\newcommand{\R}{\mathbb{R}}
\newcommand{\N}{\mathbb{N}}
\newcommand{\Z}{\mathbb{Z}}
\newcommand{\C}{\mathbb{C}}
\newcommand{\F}{\mathbb{F}}
\newcommand{\Q}{\mathbb{Q}}
\newcommand{\Tor}{\ \text{or}\ }
\newcommand{\Tand}{\ \text{and}\ }
\newcommand{\Mod}[1]{\ {\mathrm{mod}\ #1}}
\newcommand{\Pmod}[1]{\ (\mathrm{mod}\ #1)}
\newcommand{\LHS}{\text{LHS}}
\newcommand{\RHS}{\text{RHS}}
\newcommand{\cm}{\checkmark}

\DeclareMathOperator{\tr}{tr}
\DeclareMathOperator{\nullspace}{null}

\setlength{\droptitle}{-1.5cm}

\title{Understanding Analysis (Stephen Abbott) Exercises}
\author{Frank Qiang}

\begin{document}
\maketitle

\section{The Real Numbers}
\subsection{Discussion: The Irrationality of $\sqrt{2}$}
\subsection{Some Preliminaries}
\begin{tcolorbox}[title=Exercise 1.2.1.]
  \begin{enumerate}[label=(\alph*)]
    \item Prove that $\sqrt{3}$ is irrational. Does a
      similar argument work to show $\sqrt{6}$ is
      irrational?
    \item Where does the proof of Theorem 1.1.1 break
      down if we try to use it to prove $\sqrt{4}$ is
      irrational?
  \end{enumerate}
\end{tcolorbox}
\begin{proof}[Proof of (a)]
  Suppose by way of contradiction that $\sqrt{3}$ is
  rational. Then there exist $p, q \in \Z$ coprime such
  that $\sqrt{3} = p/q$. Square both sides to get
  \[{\left(\frac{p}{q}\right)}^2 = 3.\]
  Rewriting the above equality yields
  \[p^2 = 3q^2.\]
  Then $3 | p^2$, and since $3$ is prime, we also have $3 | p$. By the
  definition of divisibility there exists $k \in \Z$ such that
  $p = 3k$. Substituting into the above equation yields
  \[(3k)^2 = 9k^2 = 3q^2 \implies 3k^2 = q^2.\]
  By the same logic as above, $3 | q^2$ and $3 | q$. Thus we have
  reached a contradiction as $p$ and $q$ are coprime yet they share a
  common factor of 3. Therefore the original
  assumption must have been false and $\sqrt{3}$ must be irrational.

  A similar argument works for $\sqrt{6}$. Although $6$ is not prime, it
  is square-free, so we can still make the argument that $6 | p^2$
  implies $6 | p$.
\end{proof}

\begin{proof}[Proof of (b)]
  The proof fails when we argue that $4 | p^2$ implies $4 | p$. $p = 2$
  is a counterexample as $p^2 = 4$ is a multiple of $4$ yet $p$ is not.
\end{proof}

\begin{tcolorbox}[title=Exercise 1.2.2.]
  Show that there is no rational number $r$ satisfying $2^r = 3$.
\end{tcolorbox}
\begin{proof}
  Suppose by way of contradiction that there exists such $r \in \Q$.
  Express this $r$ as $p/q$ where $p, q \in \Z$ are coprime and $q \ne 0$.
  Then
  \[2^{p/q} = 3 \implies 2^p = 3^q.\]
  after raising both sides to the $q$-th power. And here we have reached
  a contradiction as an integer power of $2$ can never equal a non-zero
  integer power of $3$. So there cannot exist such an $r \in \Q$.
\end{proof}
\end{document}
