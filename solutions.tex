\documentclass[12pt, letterpaper]{article}
\usepackage[margin=1in]{geometry}
\usepackage{enumitem}
\usepackage{graphicx}
\usepackage{amsmath}
\usepackage{amssymb}
\usepackage{amsthm}
\usepackage{bbm}
\usepackage{bm}
\usepackage{xcolor}
\usepackage{soul}
\usepackage{adjustbox}
\usepackage{titling}
\usepackage{parskip}
\usepackage{tcolorbox}

\graphicspath{ {./images/} }

% \setlist[enumerate, 2]{label=\roman*.}
\setlist[enumerate]{nosep}
\setlist[enumerate]{topsep=0pt}

\newcommand{\R}{\mathbb{R}}
\newcommand{\N}{\mathbb{N}}
\newcommand{\Z}{\mathbb{Z}}
\newcommand{\C}{\mathbb{C}}
\newcommand{\F}{\mathbb{F}}
\newcommand{\Tor}{\ \text{or}\ }
\newcommand{\Tand}{\ \text{and}\ }
\newcommand{\Mod}[1]{\ {\mathrm{mod}\ #1}}
\newcommand{\Pmod}[1]{\ (\mathrm{mod}\ #1)}
\newcommand{\LHS}{\text{LHS}}
\newcommand{\RHS}{\text{RHS}}
\newcommand{\cm}{\checkmark}

\DeclareMathOperator{\tr}{tr}
\DeclareMathOperator{\nullspace}{null}

\setlength{\droptitle}{-1.5cm}

\title{Understanding Analysis (Stephen Abbott) Exercises}
\author{Frank Qiang}

\begin{document}
\maketitle

\section{The Real Numbers}
\subsection{Discussion: The Irrationality of $\sqrt{2}$}
\subsection{Some Preliminaries}
\begin{tcolorbox}[title=Exercise 1.2.1.]
  \begin{enumerate}[label=(\alph*)]
    \item Prove that $\sqrt{3}$ is irrational. Does a
      similar argument work to show $\sqrt{6}$ is
      irrational?
    \item Where does the proof of Theorem 1.1.1 break
      down if we try to use it to prove $\sqrt{4}$ is
      irrational?
  \end{enumerate}
\end{tcolorbox}
\begin{proof}[Proof of (a)]
  Suppose by way of contradiction that $\sqrt{3}$ is
  rational. Then there exist $p, q \in \Z$ coprime such
  that $\sqrt{3} = p/q$. Square both sides to get
  \[{\left(\frac{p}{q}\right)}^2 = 3.\]
  Rewriting the above equality yields
  \[p^2 = 3q^2.\]
\end{proof}
\end{document}
